%% LyX 2.1.4 created this file.  For more info, see http://www.lyx.org/.
%% Do not edit unless you really know what you are doing.
\documentclass[english]{article}
\usepackage[T1]{fontenc}
\usepackage[latin9]{inputenc}
\usepackage{textcomp}
\usepackage{babel}
\begin{document}

\part{Average Afflux Analysis}

In this section we want to analyse the number of all the visits people
made, during the simulation, distributed among all restaurants. In
this sense, we know that a person each turn goes to a restaurant according
to the rank-list stored in his knowledge database or chooses a random
place. Hopefully, through the duration of the entire simulation each
person will visit almost all restaurants. After a first transient
where the agent discovers the world and progressively adjourn his
knowledge database, on the effective quality of the restaurants where
he ate, we deserve that after a definite number of turns people start
becoming more incline in visiting the best restaurant they discovered
by exploring the world. In this sens, indeed, people have ascertained
the real quality of each diner by visiting it either since it was
the restaurant they thought was the best or because they went there
because at random. 

We made up this simulation by considering 8 restaurants with different
rankings uniformly distributed over the entire range, from zero to
five. Therefore, we have restaurants which ranks are:

\emph{METTERE FORMULA DELLA RELAZIONE LINEARE E POI TABELLA CON I
VALORI DEI RANK PER COMPLETEZZA. TANTO E' UNA 2X8}
\begin{itemize}
\item Restaurant 1: 0.31
\item Restaurant 2: 0.94
\item Restaurant 3: 1.56
\item Restaurant 4: 2.19
\item Restaurant 5: 2.81
\item Restaurant 6: 3.44
\item Restaurant 7: 4.06
\item Restaurant 8: 4.69
\end{itemize}
For what concernes the other parameters we have \textbf{???} people
who chooses 50\% of time the best restaurant they can otherwise they
goes randomly. Moreover, we considered 3 different cases where we
changed the maximum capacity of each restaurant in the simulation.
In the first situation the maximum capcaity is set to 5, then to 10
and the last case has restaurants with maximum 15 tables available
each. For every situation we made 5 simulations in order to have some
data on which making some statistical considerations on the emerging
behaviours. Since this, once we made the 5 simulations for each case
we mediated the results in order to have a mean tren to analyze.

In light of this, we start considering one person at once. We analyze
the percentage of time that the same person goes to different restaurants
during the entire simulation. Thus, we create an histogram were each
bar represents a restaurant (ordered on the rank from the lowest to
the highest one), while on the vertical axis we have the number of
time someone goes to the corresponding restaurant.

Since all people are similar and we have no reason to prefere an agent
(person) instead of one other, we decide to mediate above all ....\textbf{number
???..}.. people taking part of the simulation. The result of this
analysis is shown in picture number \textbf{???.}

We start by looking at results for the case with maximum evaluation
up to five. As we could have foreseen, since the capacity is low,
many people will often receive a refuse message, especially when everyone
has discovered which one is the real best restuarant and managed (when
the adopted strategy is ``chose the best'') to go there. This is
the reason why people should reconcile themselves in not going to
the restaurant they prefere and be pleased to go to a lower ranked
one. 

Besides, we observe that higher is the restaurant quality, higher
will be the number of visits it receives; we remark that in picture
number \textbf{??? }where worst ranked restaurants have a low level
of frequentation, while the afflux increases towards the maximum level,
reached for the best ranked restaurant, according to the restaurant
quality.

GRAFICO ISTOGRAMMA 1

Moreover, someone could notice that the restaurant which rank is 4.06
received an higher number of visits comparing to the 4.69 restaurant.
As we will see better in later sections, the reason of this phenomena
could be related to two main factors: first of all if we consider
a first transient period, where people do not know the real quality
of each restaurant, they would easily go to the bad restaurant since
they are convinced they are the best one. The second motivation is
linked to the choose strategy that everyone adopts on each turn; as
a matter of fact, when the 50\% of probability lead people to opt
for ``random choose'' there is no reason to go for the best restaurant.
This is the reason why even lower restaurants receives a minimum number
of visits (three in this case) during the simulation.

Looking at next simulation, maximum capacity up to 10, we notice that
the behaviours observed before come out again.

GRAFICO ISTOGRAMMA 2

Furthermore, we point out that, since the higher capacity of restaurants,
people will be refused rather than before and the trend in going to
the best restaurant, on the all lenght of the simulation, will be
more evident.

Lastly if we look and the third simulation, when capacity is up to
15 what we said before is just confirmed

GRAFICO ISTOGRAMMA 3

Nevertheless, there is one more unpredictable thing to outline. Looking
at the 3 graphs(?) we showed before, we notice that if the restaurant
rank is lower than, approximately, 2.9 it does not matter how lower
it is, that the trend will be about the same. In fact, the first 5
restaurants, for each simulation, usually always have the same number
of visits within the error. This is much more evident when we augment
the maximum capacity of the diner where the main factor responsible
of visiting bad restaurants is related to the probability of the choosing
strategy. This teach us that no matter how bad a restaurant is, it
would rather be completely void.

As for clarification, the factors that lead people to go randomly,
in real life, are sundry: for example I could think that restaurant
1 is really bad, and even if that is the truth I'm with other people
that are convinced of the contrary. Thus, if I do not want to break
up the friendship I will be obliged to resign myself and going to
a bad restaurant, following the rest of the company, even if I'm not
pleased of that. This is not a completely random choose in real life,
but could be modeled as it was in our MAS simulation.


\part{Percentage Afflux Analysis}

In this second part of the analysis we will use the same simulations
as before facing the problem from a different point of view. While
before we considered separately where a person went on each turn,
and then meadiating on all people we saw how the afflux to diners
changed through the simulation, now our aim is to shift our attention
directly on restaurants. In this sense, we now focus on estimating
what is the percentage of people afflux and how it evolves turn by
turn. As sais before we took in consideration 5 simulation for each
one of three different cases with maximum capacity set to 5, 10 and
15.

Focusing on the first situation we immediately observe that since
the capacity is low, people should even frequent bad diner. Nevertheless,
as we could expect, both the two top-ranked restaurants are always
full after turn 5. As a matter of fact, we outpointed before the importance
of the first 5 turns as a sort of transient period where people should
discover the environment updating their knowledge database.

GRAFICO TEMPO 1 + DI NUOVO ISTOGRAMMA 1 (3.1)

Moreover, if we look at both the graphs in picture \textbf{3.1???
}and we focus on the initial time steps we are now able to better
explain what we observed in the previous section. In fact, before
we noticed that the second best ranked restaurant received, on average,
more visits compared to the best one. In the first transient most
of the people are initially convinced that the second restaurant is,
on the contrary, the best one and at the beginning they all manage
to go there. Since the capacity is so low, this has an huge effect
on the mean afflux when we mediate on all people and the result is
showed in the istogram of picture \textbf{3.1???.}

As for the second simulation things are more or less the same. Since
the capacity is higher the effect of the first transient is toned
down and from the fifth turn all people manage to go to the best-ranked
restaurant. As said before, the capacity is limited so many people
will be rejected and will opt for another restaurant; furthermore,
we see again some statistical fluctuations dued both to the limited
capacity and to the possibility of going at random the 50\% of turns.

GRAFICO TEMPO 2 + DI NUOVO ISTOGRAMMA 2 (3.2)

In the third simulation with the capacity up to 15 we remark the same
trend as before\footnote{See picture 3.3}. Besides, there are two
important things to underline. 

First of all, we notice that the transient effect is much more attenuated
and, as said before, we suppose that the visits to bad restaurants
are mostly related to the strategy adopted from the agent in choosing
the diner.

Secondly, the best ranked restaurant most of the time is never fullfilled;
this could seem to be counterintuitive. Again, we could imagine that
the reason is that the capacity of the best restaurant is high enough
to host all people that decided to adopt the ``best choose (METTERE
NOME GIUSTO DELLA STRATEGIA)'' strategy while other people, who went
randomly, will book somewhere else. 

GRAFICO TEMPO 3 + DI NUOVO ISTOGRAMMA 3 (3.3)


\subsubsection*{Paragrafetto da aggiungere all'inizio della parte di analisi di Fra}

As for the analysis, once we generated and stored the .csv files,
with the result of the communications, through the log of our program
in Jade, we process them using an iPython notebook. Thus, we developped
some tools, one specifical for each type of analysis, where we red
and organised the .csv files with the \emph{Pandas} library. Then,
we used the \emph{Numpy} library in making some mathematical work,
computing standard deviations, fitting curves and so on. Lastly, we
showed our result plotting what we obtained thorugh the \emph{Matplotlib}
library and even used it to export them as a .png files.


\part{Further Considerations on Strategy's Selection}

In previous sections we interpeted part of our results saying that
some unpredictable phenomena were related to the fact that people
could adopt two different stategies in the ``choose'' phase of the
communication: one that lead them to go random and the another one,
more rational, that take them to the best restaurant. 

In order to better understand the weight that this random component
takes in our analysis, we try to remake the same simulations as before
by giving the 100\% of probability to the ``choose the best'' strategy.
In this sense, we deserve to not have statistical fluctuations and
to observe a more deterministic and rational collective behaviour.

The results are shown in the following paragraphs.

As for the first simulation we immediately notice that the histogram
which shows the average afflux is quite different than before. For
the first 2 restaurant we have a really low contribute dued to the
first transient. Since the capacity is low, the other restaurant will
be often frequent by cutomers. Naturally, people will be more pleased
to go in better diner so the average afflux increase with the rank
of the location. Besides, looking at the second graph we have the
confirmation that the trend is more deterministic. We do not have
statistical fluctuations since people always manage to opt for the
top-ranked diner which are progressively fullfilled. Thus, it is evident
that the best restaurant is immediately filled while the one which
rank is 1.56 needs about 23 turns to be always fully booked: people
realize that it is the \emph{``less worse''} place with still available
places.

ISTOGRAMMA + GRAFICO (NO RANDOM CAPIENZA 5)

In the second case we have an higher capacity for each diner so only
the first 3 restaurant will be filled. As a matter of fact, since
we have \textbf{???} people in the simulation, the place available
in three restaurants will be high enough to host all of them. The
frequentation of bad diner progressively decrease while people discover
there is something better in town. Moreover, looking at the histogram
we have the confirmation that the only contribute to worst six restaurants
is dued to the beginning ignorance of people whom knowledge database
is initialized randomly. Again, the beginning transient plays an important
role and outlines the ``intelligence'' of agents which trend is
to explore the external environment adjourning their thought.

ISTOGRAMMA + GRAFICO (NO RANDOM CAPIENZA 10)

In the third, and last, simulation things are similar again. The difference
here is that 2 restaurants with capacity up to 15 are more or less
enough to satisfy people's demand.

Moreover, it is interesting to underline that the process of discovering
th entire environment here is much more slow. As a matter of fact,
we can notice that after 30 turns there are people who still belive
the third-ranked restaurant is better than the second one since someone
still manages to go to the diner which rank is 3.44 even there's still
place available in a better restaurant. We could imagine that with
a number of turn high enough, less people will go to worse restaurant
and even the second-ranked diner will be filled as in the situations
previously showed. 

ISTOGRAMMA + GRAFICO (NO RANDOM CAPIENZA 15)


\part{Conclusions}


\part*{DA FARE:}

- SISTEMARE PARTE ANALISI INIZIALE

- AGGIUNGERE SPIEGAZIONE SU COME ABBIAMO FATTO I GRAFICI (vedi paragrafetto
sottostante)

- AGGIUNGERE DA QUALCHE PARTE FORMULE SU COME SI AGGIORNA LA VALUTAZIONE
NEL KNOWLEDE DATABASE IN EVALUATE


\section*{Commenti da fare ai comportamenti osservati}
\begin{itemize}
\item Osservo che ovviamente maggiore � la capienza dei vari ristoranti
e maggiore sar� la tendenza delle persone ad andare nel ristorante
migliore una volta che avranno esplorato la maggior parte del mondo
\item Commentare transiente iniziale in cui l'agente va alla scoperta del
mondo, molti tendono ad andare nei primi 5 turni (circa) anche nei
ristoranti scarsi ma perch� nel loro knowledge databse sono convinti
essere i migliori essendo l'inizializzazione del pensiero degli agenti
person casuale.
\item Osserviamo che ci sono diverse fluttuazioni statistiche molto evidenti
specialmente nell'ultimo grafico con capienza 15. Qui pur essendoci
posto alcune volte capita che il ristorante migliore non si riempia
e gente vada in ristoranti meno buoni. Il tutto � dovuto al fatt che
50\% delle volte vanno a caso e 50\% delle volte scelgono il migliore.
Passo successivo: fare confronto con stessa simulaazione in cui il
100\% delle volte vanno nel migliore.
\item Osservo che in tutte le situazioni c'� poca differenza tra i ristoranti
scarsi. Nel senso che se la valutazione del ristorante � inferiore
a 2.5 circa esso si ocmporter� circa allo stesso modo di quelli a
votazione pi� bassa. Per la precisione che il rank sia 0.3 o 2.1 l'affluenza
delle persone sar� circa la stessa in tutte le simulazioni (ovviamente
in proporzione alla capienza). Fenomeno tanto pi� evidente nel caso
di capienza 15. Nel caso a capienza 5 ovviamente si faranno sentire
maggiormente le fluttuazioni statistiche dovute al 50\% random e pertanto
l'istogramma sembra seguire circa una distribuzione omogenea. Mediando
su tutte le persone quindi quelli che opteranno per i ristoranti pi�
scarsi saranno quelli che non trovano posto altrove e quelli che vanno
a caso.
\end{itemize}

\subsection*{Le fluttuazioni statistiche osservate negli istogrammi sono principalmente
dovute a:}
\begin{enumerate}
\item Transiente iniziale in cui ristoranti peggiori sono pi� popolati di
quelli migliori (si vede bene nel caso a capienza 5)
\item Effeto della scelta casuale che si alterna a parit� di probabilit�
con la strategia del best choose
\end{enumerate}

\section*{Idee da inserire in ``Implementazioni future''}
\begin{enumerate}
\item Inserire vari parametri su cui basare la scelta del ristorante. Oltre
alla qualit� complessiva inserire altri gradi di liberta come ad esempio:\end{enumerate}
\begin{itemize}
\item Tipo di cucina (giapponese, italiana, pizza...)
\item Pulizia
\item Esigenze alimentari (vegani, vegetariani, fruttariani, respiriani
:D...)
\item Posizione (vicino al centro, o nel raggio di qualche km dalla posizione
-> ricerca geolocalizzata)\end{itemize}
\begin{enumerate}
\item Inserire come variabile il Prezzo, introducendolo come grado di libert�
e come parametro su cui basare eventuali nuove strategia di scelta.
\item Integrare una nuova scelta ``Weighted Random Choose'' in cui la
scelta casuale del ristorante utilizzata nella nostra simulazione
� un ibrido con la scelta del ristorante migliore. Cio� vado a caso
sempre pesando la casualit� con i miei parametri di minimum evaluation
(e.g. vado a caso ma non troppo)
\item Estendere questo tipo di modello di comunicazione anche ad altri domini
quali il booking di alberghi, la ricerca di negozi specializzati (ad
esempio abbigliamento) permettendo di fornire un servizio a 360\textdegree{}
su tutto ci� che ci viene in mente (da scrivere per bene)
\item Altre idee....\end{enumerate}

\end{document}
